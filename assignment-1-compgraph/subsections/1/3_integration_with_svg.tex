\subsection{Integration with SVG}

Scalable Vector Graphics (SVG) provides an ideal testbed for rasterization algorithms because it contains all the fundamental primitives:

\begin{itemize}
    \item \textbf{Basic shapes}: lines, rectangles, circles, polygons
    \item \textbf{Complex paths}: Bézier curves, arcs (simplified to line segments)
    \item \textbf{Styling}: stroke, fill, opacity, colors
    \item \textbf{Transformations}: translate, rotate, scale applied hierarchically
    \item \textbf{Images}: raster images embedded within vector graphics
\end{itemize}

The SVG coordinate system uses floating-point coordinates with (0,0) at the top-left, positive X rightward, and positive Y downward, which maps naturally to screen coordinates.

This foundational understanding of rasterization concepts will enable you to implement a complete 2D rendering pipeline, gaining deep insights into how graphics hardware and software systems convert mathematical descriptions of shapes into the pixels you see on screen.