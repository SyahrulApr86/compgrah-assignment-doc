\subsection{Rendering Pipeline Flow}

The complete rendering process follows this sequence:

\begin{center}
\texttt{SVG File} $\rightarrow$ \texttt{Parse Elements} $\rightarrow$ \texttt{Apply Transforms} $\rightarrow$ \texttt{Rasterize Primitives} $\rightarrow$ \texttt{Anti-alias} $\rightarrow$ \texttt{Display}
\end{center}

\subsubsection{Step 1: SVG Loading and Parsing}
The system loads an SVG file and parses it into a hierarchy of elements (lines, triangles, rectangles, images, groups).

\textbf{Implementation Status}: \textcolor{green}{\textbf{[COMPLETE]}} \textit{No student work required}

\subsubsection{Step 2: Transform Hierarchy Processing}
The renderer traverses the SVG element tree, applying nested transformations and drawing each element.

\textbf{Student Task}: \textcolor{red}{\textbf{[TODO]}} \textbf{Implement \texttt{draw\_element()}} - Apply hierarchical transformations and render elements based on their type.

\subsubsection{Step 3: Viewport and Coordinate Transformation}
Convert from SVG world coordinates to screen pixel coordinates using viewport transformations.

\textbf{Student Task}: \textcolor{red}{\textbf{[TODO]}} \textbf{Implement \texttt{get\_matrix()}} - Handle pan/zoom operations and coordinate space conversions.

\subsubsection{Step 4: Primitive Rasterization}
Convert geometric primitives (lines, triangles) into discrete pixels.

\textbf{Student Tasks}:
\begin{itemize}
    \item \textcolor{red}{\textbf{[TODO]}} \textbf{Implement \texttt{rasterize\_line()}} - Bresenham's algorithm for line drawing
    \item \textcolor{red}{\textbf{[TODO]}} \textbf{Implement \texttt{rasterize\_triangle()}} - Edge function-based triangle filling
\end{itemize}

\subsubsection{Step 5: Image and Texture Processing}
Rasterize image elements by mapping screen pixels to texture coordinates and sampling colors.

\textbf{Student Tasks}:
\begin{itemize}
    \item \textcolor{red}{\textbf{[TODO]}} \textbf{Implement \texttt{rasterize\_image()}} - UV mapping and inverse transformation
    \item \textcolor{red}{\textbf{[TODO]}} \textbf{Implement texture sampling functions}:
    \begin{itemize}
        \item \texttt{sample\_nearest()} - Nearest neighbor filtering
        \item \texttt{sample\_bilinear()} - Bilinear interpolation
        \item \texttt{sample\_trilinear()} - Trilinear filtering with mipmaps
        \item \texttt{generate\_mips()} - Mipmap generation using box filtering
    \end{itemize}
\end{itemize}

\subsubsection{Step 6: Anti-aliasing Pipeline}
Use supersampling to reduce aliasing artifacts by rendering at higher resolution and downsampling.

\textbf{Student Tasks}:
\begin{itemize}
    \item \textcolor{red}{\textbf{[TODO]}} \textbf{Implement \texttt{fill\_sample()}} - Write individual samples to high-resolution buffer
    \item \textcolor{red}{\textbf{[TODO]}} \textbf{Implement \texttt{fill\_pixel()}} - Fill all samples belonging to a pixel
    \item \textcolor{red}{\textbf{[TODO]}} \textbf{Implement \texttt{resolve()}} - Downsample from sample buffer to pixel buffer using box filter
\end{itemize}

\subsubsection{Step 7: Alpha Compositing}
Blend transparent and semi-transparent colors using alpha channel information.

\textbf{Student Task}: \textcolor{red}{\textbf{[TODO]}} \textbf{Implement \texttt{alpha\_blend()}} - Standard alpha blending formula for transparency effects.

\subsubsection{Step 8: Display}
Convert the final pixel buffer to a format suitable for display in the GUI.

\textbf{Implementation Status}: \textcolor{green}{\textbf{[COMPLETE]}} \textit{No student work required}
