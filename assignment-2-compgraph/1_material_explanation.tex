% Material Explanation Section for SVG Rasterizer Assignment
% This content should be inserted into the main.tex file

\section{Introduction}

This assignment focuses on implementing a \textbf{Halfedge Mesh data structure} and several \textbf{mesh subdivision algorithms}, which are fundamental techniques in computer graphics for representing and refining polygonal surfaces.  
Unlike simple vertex-face lists, the halfedge representation explicitly encodes mesh connectivity, making it efficient to traverse adjacency relationships (vertices, edges, and faces).  

On top of this structure, students will implement several algorithms such as \textbf{Linear Subdivision}, \textbf{Triangulation}, \textbf{Loop Subdivision}, and \textbf{Catmull--Clark Subdivision}, enabling the conversion of coarse polygonal models into smoother and more detailed meshes.  
Through this assignment, you will gain a deeper understanding of mesh topology, geometry processing, and the foundations of modern surface modeling.  

\begin{figure}[H]
    \centering
    \includegraphics[width=0.75\linewidth]{images/subdivision.png}
    \caption{Subdivision example}
    \label{fig:subdivision}
\end{figure}

\subsection{What is Halfedge Mesh?}

A halfedge mesh is a data structure used to represent polygonal meshes in computer graphics and geometry processing. 
Unlike simple adjacency lists or winged-edge representations, the halfedge structure explicitly encodes connectivity 
by splitting each edge into two oppositely directed halfedges.  

\subsubsection{Motivation for Halfedge Representation}

\subsubsection{Basic Elements}

\subsubsection{Mesh Traversal}

\subsection{Fundamental Mesh Geometry}

...

\subsubsection{Centroid (Face Point)}

\subsubsection{Edge Midpoint (Edge Point)}

\subsubsection{Vertex Valence}

\subsubsection{Normal \& Area}

\subsection{Subdivision Algorithms}

...

\subsubsection{Linear Subdivision}

\subsubsection{Loop Subdivision (Triangular Meshes)}

\subsubsection{Catmull--Clark Subdivision (General Meshes)}


\subsection{Topology and Adjacency}

...

\subsubsection{Edge--to--Faces Adjacency}

\subsubsection{Vertex-to-Faces, Edges, Neighbors}
